% !TEX encoding = UTF-8 Unicode 

\chapter{Przegląd istniejących rozwiązań}

W tym rozdziale omówiono literaturę i istniejące rozwiązania problemu rozwiązywanego w dalszej części tej pracy. Pod uwage brano prace, które korzystają z klasycznych metod uczenia maszynowego oraz metod zawierających elementy sieci neuronowych. Analizie zostaną poddane użyte algorytmy jak i podejścia do osiągnięcia celu wspomagania decyzji automatyki domowej.


\section{Sytemy korzystające z wykrywania epizodów}
Episode Discovery (wykrywanie epizodów) to metoda data miningu (kopania danych) wykorzystująca istniejący ciąg występujących po sobie wydarzeń do wykrywania w nich powtarzalnych znaczących epizodów. Wśród epizodów można wyróżnić, tak zwane, epizody znaczące, które według zależnych od algorytmu charakterystyk, określają dany epizod jako często występujący.

Wykorzystywanie algorytmów wykrywania epizodów na zapisanych już strumieniach wydarzeń pozwala na znalezienie pewnych nawyków i zależności z codziennego korzystania z domowych urządzeń. Znalezione i wyodrębnione epizody mogą zostać użyte z innymi algorytmami w celu podniesienia ich celności. Tak przetworzone dane wejściowe z dodatkowym użyciem innego algorytmu dają zdecydowanie lepsze wyniki niż w wypadku użycia samych sieci neuronowych bądź samego wykrywania epizodów \cite{episode_discovery_1}, \cite{episode_discovery_2}. 

% Smart Home Automation using IoT and Deep Learning 
% An Intelligent, Secure, and Smart Home Automation System
% Home Automation System Using IoT and Machine Learning Techniques
% Use of Prediction Algorithms in Smart Homes
% Smart Home Automation Using Machine Learning Algorithms
% A machine learning approach to predict the activity of smart home inhabitant
% Improving Home Automation by Discovering Regularly Occurring Device Usage Patterns

Ważnym elementem wykorzystania technik wykrywania epizodów jest prawidłowy wybór hiperparametrów algorytmu jak i samego algorytmu, ponieważ inne wartości parametrów wybierających epizod znaczący może mocno wpływać na końcowy wynik \cite{episode_discovery_2}, a co za tym idzie, na końcową jakość predykcji akcji użytkownika.


\section{Systemy korzystające z Modeli Markowa}

% An Intelligent, Secure, and Smart Home Automation System
% MavHome: An Agent-Based Smart Home

\section{Systemy korzystające z SVM}

% MavHome: An Agent-Based Smart Home


% Tu jest ładnie napisane dużo: Smart Home Automation using IoT and Deep Learning