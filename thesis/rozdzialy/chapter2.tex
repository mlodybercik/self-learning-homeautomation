% !TEX encoding = UTF-8 Unicode 

\chapter{Przegląd literatury}

% W tym rozdziale omówiono literaturę i istniejące rozwiązania problemu rozwiązywanego w dalszej części tej pracy. Pod uwage brano prace, które korzystają z klasycznych metod uczenia maszynowego oraz metod zawierających elementy sieci neuronowych. Analizie zostaną poddane użyte algorytmy jak i podejścia do osiągnięcia celu wspomagania decyzji automatyki domowej.

W tym rozdziale omówiono umiejscowienie tematu w literaturze. W pierwszej części opisano istniejące algorytmy i rozwiązania problemu. Pod uwagę brano te metody, które korzystają z klasycznych metod uczenia maszynowego oraz te, które zawierają elementy sieci neuronowych. Analizie zostaną poddane użyte algorytmy, jak i podejścia do osiągnięcia celu wspomagania decyzji automatyki domowej. W drugim etapie zawarte zostały opisy budowy i sposobów działania urządzeń Internetu Rzeczy.

\section{Algorytmy i rozwiązania}
\subsection{Systemy korzystające z algorytmów wykrywania epizodów} \label{subsec:episode_discovery}
\textit{Episode discovery} (ang. wykrywanie epizodów) to metoda \textit{data miningu} (ang. kopania danych) wykorzystująca istniejący ciąg występujących po sobie wydarzeń do wykrywania w nich powtarzalnych znaczących epizodów. Wśród epizodów można wyróżnić tak zwane epizody znaczące, które według zależnych od algorytmu charakterystyk, określają dany epizod jako często występujący.

Czesto można spotkać się z pewną interpretacją ciągu zdarzeń w systemie, gdzie każda występująca po sobie w pewnym oknie czasowym akcja jest reprezentowana jako odpowiedni znak ze zbioru możliwych zdarzeń. W przypadku opisu stanu systemu za pomocą reprezentacji znakowej, korzysta się z metodyki, gdzie duża litera oznacza przejście stanu danego urządzenia w załączone, a mała litera oznacza wyłączenie. W przykładowych ciągach zdarzeń, \verb|baDgb|, \verb|abD|, \verb|Dagb|, można zauważyć, że zawsze po wystąpieniu wyłączenia urządzenia \verb|a|, następuje wyłączenie urządzenia \verb|b|. W rzeczywistości algorytmy są bardziej zaawansowane, ponieważ są w stanie wykryć epizody znaczące, które nie występują zawsze a w pewnym odsetku wszystkich wydarzeń i ponadto są w stanie operować na dużo dłuższych łańcuchach historycznych.

Wykorzystywanie algorytmów wykrywania epizodów na zapisanych już strumieniach wydarzeń pozwala na znalezienie pewnych nawyków i zależności z codziennego korzystania z domowych urządzeń. Znalezione i wyodrębnione epizody mogą zostać użyte z innymi algorytmami w celu podniesienia ich celności. Tak przetworzone dane wejściowe z dodatkowym użyciem innego algorytmu dają zdecydowanie lepsze wyniki niż w wypadku użycia samych sieci neuronowych bądź samego wykrywania epizodów \cite{episode_discovery_1}, \cite{episode_discovery_2}. 

% Smart Home Automation using IoT and Deep Learning 
% An Intelligent, Secure, and Smart Home Automation System
% Home Automation System Using IoT and Machine Learning Techniques
% Use of Prediction Algorithms in Smart Homes
% Smart Home Automation Using Machine Learning Algorithms
% A machine learning approach to predict the activity of smart home inhabitant
% Improving Home Automation by Discovering Regularly Occurring Device Usage Patterns

% to check: dodać przypis na dole, że to są przykładowe algorytmy
Ważnym elementem wykorzystania technik wykrywania epizodów jest prawidłowy wybór algorytmu (takiego jak np. SPADE \cite{SPADE}, SPEED \cite{SPEED}, WINEPI \cite{WINEPI}, ...) jak i jego własności, ponieważ inne wartości parametrów wybierających epizod znaczący mogą mocno wpływać na końcowy wynik \cite{episode_discovery_2}. O ile w przypadku algorytmu, nieprawidłowy wybór może ograniczać się do wydłużonego czasu poszukiwania epizodów, a co za tym idzie, większego zużycia zasobów przez system, tak w przypadku niewłaściwych parametrów, algorytm może proponować za dużo akcji i nawyków, które źle będą wpływały na końcowy wynik, co w końcu sprawi, że komfort korzystania z takiego systemu będzie niedostateczny.

% to check: tutaj jest coś nie tak w pierwszym zdaniu. (np. dzien tygodnia, pogoda)
% Podejście wykrywania epizodów jest metodą skupiającą się na pewnych ciągach zdarzeń, które nie biorą pod uwagę żadnych zewnętrznych parametrów. 
Podejście wykrywania epizodów jest metodą, która skupia się na ciągach zdarzeń i ich wzajemnej kolejności, a nie na konkretnym stanie systemu. Sprawia to, że system uczy się ciągów wydarzeń nie biorąc pod uwagę aktualnego stanu, czyli na przykład pory dnia, temperatury w pomieszczeniu, dnia tygodnia czy też pogody. Połączenie klasycznych algorytmów \textit{episode discovery} z dodatkowymi algorytmami, które biorą pod uwagę inne parametry systemu, rozwiązuje ten problem, ale dodaje dużo skomplikowania w implementacji.

Korzystanie z metody dużych i małych liter do oznaczania wyłączeń i włączeń urządzeń tworzy dodatkowo pewne ograniczenia w postaci braku lub bardzo skomplikowanej obsługi urządzeń o niebinarnych stanach. O ile w przypadku urządzeń gdzie stanów jest kilka, jak na przykład systemy HVAC, można każdy z trybów pracy zinterpretować jako inną czynność, tak w przypadku urządzeń gdzie istnieje teoretycznie nieskończenie wiele stanów pośrednich, tak jak na przykład w ściemniaczach żarówkowych, nie jest możliwe reprezentowanie każdego z nich.

% \subsection{Systemy grafowe}
\subsection{Systemy korzystające z procesów Markowa}
Podczas modelowania zmian stanu systemu automatyki domowej można skorzystać z podejścia, gdzie próbujemy opisać ciąg zdarzeń za pomocą prawdopodobieństwa przejść między stanami. Bardzo pomocne w takiej strategii okazuje się korzystanie z modeli Markowa. Istnieje pewne rozszerzenie, które okazuje się bardziej pomocne w wypadku modelowania ludzkiej interakcji i zachowań z powodu uwzględnienia niezależnych i nieznanych przez model stanów, nazywane ukrytym modelem Markowa. Czyste podejście z prawdopodobieństwem odpowiada na pytanie, jaka jest szansa na wykonanie akcji \verb+A+ pod warunkiem tego, że poprzednio wykonana akcja to \verb+B+, dodatkowe ukryte informacje pomagają w dokładniejszym określeniu przewidywanego stanu systemu.

Podobnie jak w przypadku wykrywania epizodów (\ref{subsec:episode_discovery}) do wytworzenia reprezentacji zmian systemu wykorzystywany jest literowy zapis, co sprawia, że te podejścia mają takie same ograniczenia. Jedna z prac \cite{markov_1}, próbuje rozwiązać problem związany z rzeczywistą (niebinarną) naturą pewnych stanów poprzez kwantyzację w konkretne stany. Konkretne wartości temperaturowe zamieniane są na arbitralny identyfikator wskazujący na daną temperaturę, co używane jest w modelu jako jeden z parametrów.

Podejście znalezione w \cite{markov_1} próbuje rozwiązać, za pomocą autorskiej metody \textit{IHMM} (Improved Hidden Markov Models), problem powiązania pewnych konkretnych nawyków i ciągów wydarzeń z pewną temporalną zmienną, co sprawia, że system podpowiada konkretne akcje dokładniej o konkretnych porach dnia, lecz dalej nie rozwiązuje problemu rozpoznania np. dni tygodnia czy pogody.

Inna implementacja \cite{markov_2}, korzysta z innego, również autorskiego, rozwinięcia modeli Markowa \textit{TMM} (Task-based Markov Model). Metoda skupia się na zidentyfikowaniu wysokopoziomowych zadań, które to dalej są wykorzystywane do stworzenia pomniejszych modeli reprezentujących ciąg zdarzeń w konkretnym zadaniu. Zadania są identyfikowane na podstawie przerwy pomiędzy kolejnymi wydarzeniami oraz na podstawie dodatkowej informacji o lokalizacji danego urządzenia. Operacje wraz z informacjami o ilości, długości i położenia w przestrzeni używane są przez algorytm k-średnich do pogrupowania zadań i stworzenia zbioru konkretnych ciągów.

% TEORETYCZNE PODSTAWY ZASTOSOWAŃ UKRYTEGO MODELU MARKOWA DO ROZPOZNAWANIA WZORCÓW
% MavHome: An Agent-Based Smart Home
% Behavior prediction using an improved Hidden Markov Model to support people with disabilities in Smart Homes

% inne:
% MavHome: An Agent-Based Smart Home
% An Intelligent, Secure, and Smart Home Automation System
% Tu jest ładnie napisane dużo: Smart Home Automation using IoT and Deep Learning

\subsection{Sieci neuronowe}

% Automated Smart Home Controller Based on Adaptive Linear Neural Network 
% Smart Home Automation-Based Hand Gesture Recognition Using Feature Fusion and Recurrent Neural Network

% to check: dopisać teorii tutaj troche. czym są głębokie sieci neuronowe
Sieci neuronowe to matematyczny model stosowany do celów przetwarzania informacji i podejmowania decyzji. Ich budowa, w pewnym stopniu, jest wzorowana na rzeczywistym funkcjonowaniu biologicznego systemu nerwowego. Składają się one ze sztucznych neuronów. Każdy z nich posiada parametry, które opisują jak jego aktywacje wpływają na wszystkie inne wyjścia w całej sieci. W procesie uczenia, czyli propagacji wstecznej, każdy parametr neuronu jest zmieniany w celu zminimalizowania błędu między oczekiwanym a rzeczywistym wynikiem \cite{ksiazka_tf}. Sieć nazywana jest głęboką jeśli zawiera wiele połączonych ze sobą warstw. W rzeczywistości wyróżnia się konkretne architektury w zależności od dziedziny problemu, są to między innymi sieci rekurencyjne i konwolucyjne. W pierwszej z nich odpowiednia struktura sprawia, że sygnał wyjściowy trafia ponownie na wejście sieci generując ciąg nowych próbek. Drugi rodzaj skupia się na matematycznym procesie konwolucji macierzy do przetwarzania i ekstrakcji cech.

Często spotyka się w parze z innym algorytmem sieci neuronowe, ponieważ są w stanie poprawić zdecydowanie celność predykcji niskim kosztem \cite{episode_discovery_1}. Istnieją pewne rozwiązania, gdzie sieci neuronowe są głównym sposobem na przeprowadzanie predykcji. Ich szerokie zastosowanie i wiele istniejących rozwiązań sprawia, że są w stanie występować na prawie każdej platformie obliczeniowej. Dodatkowo w ciągu ostatniej dekady widać było zdecydowany rozrost zastosowań i nowych rozwiązań korzystających w pewien sposób z głębokich sieci neuronowych.

% to check: wytłumaczyć czym jest LSTM
Różne istniejące rozwiązania i modele głębokich sieci neuronowych dają duże pole do ich zastosowań. W przypadku pracy autorstwa S. Umamageswari \cite{neural_1} zastosowano jedno z popularnych w ostatnim czasie rozwiązań \textit{LSTM} (\textit{long short-term memory}) będące rozszerzeniem warstw typu rekurencyjnego. Są one w stanie modelować pewne zależności w danych, występujące z dużym odstępem czasowym. Jest to dobre zastosowanie dla sieci wykorzystujących dużą ilość danych wejściowych zawierające informacje temporalne w odpowiedniej postaci.

Ciekawym podejściem okazuje się wykorzystanie konwolucyjnych sieci neuronowych w przypadku wprowadzania dodatkowych zmiennych do systemu \cite{conv_1}. W przytoczonej pracy system w celu określenia następnej akcji którą ma podjąć bierze pod uwagę wyraz twarzy użytkownika. Obraz twarzy, przekształcany jest przez sieci konwolucyjne do zestawu informacji mówiącym o samopoczuciu użytkownika.

% to check: "bardzo" głupio brzmi
\subsection{Podsumowanie i wnioski}
Na podstawie przeprowadzonej analizy można zauważyć, że wstępne przetwarzanie danych i odpowiednie ich spreparowanie jest kluczowym aspektem w realizacji procesu. Prawidłowa reprezentacja stanu w systemie jest kluczowa, ponieważ to na podstawie tych wartości system będzie przeprowadzał swoje predykcje. W przypadku tradycyjnych algorytmów uczenia maszynowego zauważyć można ich zdecydowane ograniczenia, których rozwiązanie wymaga zdecydowanego komplikowania problemu poprzez wprowadzanie kwantyzowanych informacji czy zmiennych reprezentujących wartości temporalne. Metodą dającą największą elastyczność okazuje się wykorzystanie głębokich sieci neuronowych, ze względu na ich możliwości, przy wcześniejszym odpowiednim przekształceniu wejścia lub wyjścia, wprowadzeniu danych o różnych formatach czy sterowaniu szerokim zakresem różnych urządzeń.

\section{Urządzenia IoT}

% A Survey based on Smart Homes System Using Internet-of-Things 
% Energy-Efficient System Design for IoT Devices
% An Overview of IoT Based Smart Homes
% A survey of Internet-of-Things: Future Vision, Architecture, Challenges and Services 
% What is a smart device? - a conceptualisation within the paradigm of the internet of things
% Impact of IOT in Current Era

% to check: nie tłumacze nigdzie czym jest urządzenie IoT, kompletnie jest do wyjebania/zmiany. nie ma nic o IoT
Internet jest zbiorem technologii który bardzo zmienił sposób funkcjonowania ludzi. Wraz ze spadającym kosztem podłączenia do Internetu, znajdujemy coraz to nowsze zastosowania dla właśnie takich systemów. Internet Rzeczy (\textit{Internet-of-Things}) to zbiór urządzeń wyposażonych w czujniki oraz odpowiednie oprogramowanie umożliwiające im zbieranie, przesył, interpretację danych oraz reagowanie. Koncepcja od momentu narodzin, na przełomie XX i XXI, została nazwana najbardziej wpływową technologią aktualnej epoki. Ludzkość jest aktualne świadkiem rewolucji w której urządzenia stale \textit{on-line} przenikają do naszego codziennego życia, gdzie poprawiają jakość i komfort ludzkiego życia \cite{iot_improves_lifes}. Plastyczność rozwiązań systemów IoT gwarantuje im obecność w różnych obszarach, takich jak zdrowie, przemysł, transport czy rolnictwo. Jednym z zastosowań, które cieszy się obecnie dużą popularyzacją wśród urządzeń IoT jest rozwiązanie \textit{smart-home}, ze względu na ich możliwości. W środowisku domowym pełnią rolę różnych czujników i siłowników, które zapewniają reszcie systemu informacji w celu przeprowadzania automatyki. Każde urządzenie, niezależnie od środowiska, składa się z dwóch nierozłącznych elementów. Są to osobno ,,rzecz'' i osobno ,,internet''.

\subsection{\textit{Rzecz}}
Kontekst rzeczy w domenie Internetu Rzeczy jest bardzo rozmyty, ze względu na swoją logiczną strukturę jak i fizyczne zastosowanie, co dalej tworzy podział na \textit{smart} i nie-\textit{smart} urządzenia. Literatura wyróżnia trzy elementy które definiują typ urządzenia. Są to: świadomość kontekstu, autonomia i łączność \cite{smart_dumb_devices}. Mowa tutaj o bardzo różnych systemach, które mogą agregować w jedność wiele czujników, wiele siłowników, procesorów i sposobów komunikacji. Do pierwszej z tych grup zaliczają się takie obiekty które, poza zbieraniem danych, są w stanie reagować na ich zmianę. Są to na przykład \textit{smart} termostaty, które na podstawie temperatury zadanej i mierzonej decydują o odpowiedniej zmianie w swoim systemie. Mierzona temperatura może pochodzić wprost z urządzenia lub też z innych urządzeń znajdujących się w środowisku. Do drugiej grupy, kontynuując poprzedni przykład, zaliczyć można urządzenia-satelity, które mają dwa główne zadania. Zadaniem pierwszego typu jest zagregowanie danych pochodzących ze środowiska i przesłanie do jednostki zarządczej. W drugim typie zadań, przyrządy reagują na polecenia pochodzące z systemu i sterują mechanizmami w celu osiągnięcia ogrzewania lub chłodzenia, dodatkowo zgłaszając informację o powodzeniu akcji.


% to check: zmienić słowa urządzenia, dopisać bo jest do wyjebania
\subsection{\textit{Internet}}
Internet w kontekście inteligentnych urządzeń to zbiór technologii i protokołów używanych przez każdy system w celu komunikacji. Bardzo ważnym aspektem w roli inteligentnych mechanizmów jest wymiana informacji, a bez tego byłoby trudno powiedzieć o rzeczywiście mądrym rozwiązaniu. Ważnym aspektem w przesyle danych, jest struktura każdej wiadomości. Systemy IoT będące rozwinięciem istniejących technologii korzystają z gotowych rozwiązań do celów komunikacyjnych. Pozwala im to na użycie istniejącej już infrastruktury do porozumiewania się miedzy sobą  \cite{rfc791}, a w przypadku braku takowej, za pomocą odpowiednich standardów spontanicznych \cite{spontaneous_networking}, \cite{spontaneous_wireless_networking}. Takie użycie protokołów pozwala systemom na integrację w prawie każdym środowisku oraz komunikację z każdym innym podłączonym do sieci obiektem.

% \subsection{\textit{Internet Rzeczy}}

% Jesteśmy w stanie w każdym momencie, gdziekolwiek jesteśmy wyszukać odpowiedzi na każde nasze pytanie.
% Gigantyczne ilości informacji sa generowane, agregowane, przetwarzane i przesyłane w każdym momencie.
% jak na przykład redukcja zużycia energii elektrycznej, czy wykonywanie pewnych czynności za użytkownika.

% badania monoselekcyjne