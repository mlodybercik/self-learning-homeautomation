\chapter{Wymagania projektowe}
\label{chapter:wymagania}
Celem tego projektu jest wytworzenie systemu zdolnego do wspomagania użytkownika w eksploatacji inteligentnych urządzeń IoT znajdujących się w jego domowym środowisku poprzez wykonywanie pewnych czynności za użytkownika. 

% tego nie wiem czy nie bede opisywał podczas implementacji
Automatyzacja nie powinna wykonywać się wprost, tj. bez ludzkiej interakcji. Użytkownik korzystający z tego systemu będzie spełniał kluczową rolę, ponieważ wymaga się, aby system dokańczał powtarzalne ciągi zadań wykonywanych przez użytkownika.

Wymagane jest aby rozpoczęcie używania tego systemu wymagała od użytkownika minimum konfiguracji, system powinien pobierać dane z istniejącego już systemu automatyki domowej i używać ich w celu wytworzenia reguł. Dodatkowo, sam system powinien być niezależny od architektury komputera na którym się znajduje. Zaleca się, aby moduł współpracował z gotowymi systemy automatyki domowej. W wypadku utraty przez komputer obsługujący system prądu, system powinien wznowić swoją pracę bez ponownego procesu tworzenia reguł decyzyjnych.

Wymaga się, aby system nie wymagał dużej ilości prac w celu dodania nowych typów źródeł i typów ujść danych, tak aby ewentualne dodawanie obsługi urządzeń dla których nie istnieje gotowe wsparcie było najprostsze.

% generalnie to to sobie zostawie pod znakiem zapytania bo nie wiem czy mi sie bedzie chciało to implementować jeszcze xD
W wypadku innej odpowiedzi systemu niż użytkownik oczekuje, użytkownik powinien mieć łatwy sposób na cofnięcie wykonania błędnej akcji do poprzedniego stanu.

% Teoretycznie moge to wciepać w tabelke
Biorąc powyższe pod uwagę, można określić listę wymagań funkcjonalnych i niefunkcjonalnych tego systemu.
Wymagania funkcjonalne:
\begin{itemize}
    \item wspomaganie codziennej eksploatacji urządzeń w domu,
    \item cofanie akcji,
    \item odczyt oraz zapis do pamięci stałej,
    \item wsparcie istniejącego systemu automatyki,
\end{itemize}
Wymagania niefunkcjonalne:
\begin{itemize}
    \item minimum konfiguracji,
    \item działanie niezależne od architektury komputera,
    \item łatwość w tworzeniu rozszerzeń,
\end{itemize}

