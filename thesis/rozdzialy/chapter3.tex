\chapter{Wymagania projektowe}
\label{chapter:wymagania}
Celem tego projektu jest wytworzenie systemu zdolnego do wspomagania użytkownika w eksploatacji inteligentnych urządzeń IoT znajdujących się w jego domowym środowisku poprzez wykonywanie pewnych czynności za użytkownika. 

% tego nie wiem czy nie bede opisywał podczas implementacji
% to check: wyjebać tych powtórzeń tyle xD
Automatyzacja nie powinna wykonywać się wprost, tj. bez ludzkiej interakcji. Użytkownik korzystający z tego modułu będzie spełniał kluczową rolę, ponieważ wymaga się, aby system dokańczał powtarzalne ciągi zadań wykonywanych przez użytkownika.

% aparat
Konieczne jest aby rozpoczęcie użytkowania tego systemu wymagało od użytkownika minimum konfiguracji. System powinien pobierać dane z istniejącego już programu nadzorczego automatyki domowej i używać ich w celu wytworzenia reguł. Dodatkowo, sam powinien być niezależny od architektury komputera na którym się znajduje. Zaleca się, aby moduł współpracował z gotowymi systemami automatyki domowej. W wypadku utraty przez komputer zasilania, system powinien wznowić swoją pracę bez ponownego procesu tworzenia reguł decyzyjnych.

Konieczne jest, aby implementacja nie wymagała dużej ilości prac w celu dodania nowych typów źródeł i ujść danych, tak aby ewentualne dodawanie obsługi urządzeń dla których nie istnieje gotowe wsparcie było najprostsze. System powinien móc wykorzystywać dodatkowe źródła danych w celach predykcji.

% generalnie to to sobie zostawie pod znakiem zapytania bo nie wiem czy mi sie bedzie chciało to implementować jeszcze xD
W wypadku innej odpowiedzi systemu niż użytkownik oczekuje, użytkownik powinien mieć łatwy sposób na cofnięcie wykonania błędnej akcji do poprzedniego stanu.

% Teoretycznie moge to wciepać w tabelke
Biorąc powyższe pod uwagę, można określić listę wymagań funkcjonalnych i niefunkcjonalnych tego systemu.
Wymagania funkcjonalne:
\begin{itemize}
    \item wspomaganie codziennej eksploatacji urządzeń w domu,
    \item cofanie akcji,
    \item odczyt oraz zapis do pamięci stałej,
    \item wsparcie istniejącego systemu automatyki,
    \item dodawanie innych źródeł danych.
\end{itemize}
Wymagania niefunkcjonalne:
\begin{itemize}
    \item minimum konfiguracji,
    \item działanie niezależne od architektury komputera,
    \item łatwość w tworzeniu rozszerzeń.
\end{itemize}

