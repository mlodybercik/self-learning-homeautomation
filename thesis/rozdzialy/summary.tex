% !TEX encoding = UTF-8 Unicode 

\chapter*{Podsumowanie}
\section*{Zrealizowana praca}
W ramach tej pracy, przeanalizowano i omówiono zagadnienie związane z automatyką domową. Przejrzano i skategoryzowano istniejące rozwiązania problemu pracy zawarte w literaturze. Każde z podejść zostało poddane analizie w celu rozpoznania mocnych i słabych stron rozwiązania. W dalszej części zagłębiono się w problem zastosowań urządzeń IoT do celów automatyki. Na podstawie literatury i istniejących rozwiązań stworzono zakres wymagań jakie powinien spełniać działający system w celu wspierania użytkownika w codziennych czynnościach domowych. Stworzone wymogi -- funkcjonalne i niefunkcjonalne -- pomogły w określeniu technologii oraz całego środowiska aplikacji w którym stworzony moduł pracuje. Wykorzystanie konkretnych aplikacji, modułów i bibliotek ukształtowało architekturę logiczną samego modułu. Korzystając z planów i architektury sporządzono w pełni funkcjonujące rozwiązanie. W celu sprawdzenia jakości systemu przeprowadzono badania i opisano wyniki.

\section*{Cel pracy}
Cel pracy został osiągnięty. Sporządzono moduł samouczący, który wykonuje epizody nauczonych akcji za użytkownika. Dodatkowo w celu integracji z rzeczywistym środowiskiem, rozwiązanie współpracuje z oprogramowaniem HomeAssistant w celu sterowania urzadzeniami IoT w domu. Zastosowanie paradygmatu zorientowanego na obiekty do implementacji modułu ułatwiło integrację systemu w dynamiczną całość i pozwoli na dalszy rozwój aplikacji. System podczas pierwszego uruchomienia stworzy modele sieci neuronowych i nauczy je na podstawie danych historycznych. W trybie bieżącej pracy, system będzie przewidywał następne akcje użytkownika i w wypadku wspólnego zamiaru, wykonywał pozostałe czynności.

\section*{Sugestie dalszych prac}
Każde stworzone oprogramowanie nigdy nie jest skończone, jest ono jedynie publikowane. Każda praca zostawia pole do dalszych usprawnień i modyfikacji. Opisany w tej pracy moduł samouczący nie jest wyjątkiem w tej zasadzie. Jedną dużą zmianą, która mogłoby poprawić jakość działania całego systemu, byłoby dynamiczne tworzenie różnych struktur sieci neuronowych w zależności od docelowego urządzenia. Usprawnienie miałoby na celu określenie dokładnej definicji ostatnich warstw w modelu, co poprawiłoby jakość połączenia między konwerterami a sercem modułu. Dodatkowo w celu poprawy jakości działania całego systemu, należałoby rozważyć inny algorytm generowania dodatkowych danych oraz ziarnisty sposób określenia dla każdego urządzenia parametrów uczenia, warstw sieci i generowania epizodów. Polem do rozważań jest problem identyfikacji i ewentualnego interweniowania w przypadku gdy dwa podobne przejścia stanów wykonują się pod warunkiem tego samego stanu systemu. 

Dalsze prace są niewątpliwym efektem tworzenia rozszerzalnego i wolnego oprogramowania. Stworzony system wspierajacy tworzenie nowych, dodatkowych urządzeń jest jedynie zachętą dla użytkownika aby ten moduł rozszerzać.

\section*{Wnioski}
Połączenie ze sobą HomeAssistant i AppDaemon to bardzo silne połączenie, w tym przypadku wzmacniane jeszcze bardziej przez uczenie maszynowe. Daje ono możliwość połączenia wszystkiego co użytkownik jest w stanie wyrazić oprogramowaniem, ze światem rzeczywistym, a dokładniej, jego środowiskiem domowym. Tworzy to perspektywę stworzenia czegoś większego niż inteligentny dom. Daje narzędzia do wykreowania czegoś większego, co świadomie będzie opiekowało się swoimi domownikami. Takie zastosowanie sterowania domem i automatyką, tworzy możliwość reagowania środowiska domowego na wydarzenia dla niego zewnętrznych, do których nie zostało specjalnie przygotowane. 

Ta praca to tylko kropla w morzu możliwości jakie taka kombinacja oprogramowania jest w stanie osiągnąć i kroplą w oceanie możliwości wykorzystania systemów IoT.
