\chapter{Implementacja}

\section{Architektura}
Na tym stadium pracy bardzo ważne jest dokładne zbadanie wymogów znajdujących się w rozdziale \ref{chapter:wymagania}, w celu odpowiedniego przygotowania architektury systemu. Błędny wybór, może skutkować ograniczeniami co do zastosowań tego systemu.

\subsection{HomeAssistant -- HA}
Jako system automatyki domowej, kontrolujący wymianę informacji naszego modułu ze światem rzeczywistym wybrano platformę HomeAssistant. Jest to jeden z największych niekomercyjnych systemów tego typu. Jego przewagą w porównaniu do niektórych innych dostępnych na rynku systemów są rzesze fanów i majsterkowiczów, którzy oferują świetne wsparcie i pomagają w rozwiązywaniu problemów. HA skupia się także na prywatności. Programiści open-source jak i użytkownicy systemu zalecają utrzymywanie systemu na swojej własnej infrastrukturze w domu. Dzięki wiernym fanom i samej architekturze projektu system posiada bardzo bogatą bibliotekę integracji z różnymi systemami, od systemów typu Google Assistant w celu dodawania funkcjonalności sterowania głosem, przez własnościowe systemy zarządzania oświetleniem w domu aż po wsparcie dodatkowych bezprzewodowych protokołów komunikacyjnych przeznaczonych do zastosowań domowych. Bardzo ważnym atutem poza szerokim polem różnych integracji jest także system rozszerzeń, gdzie użytkownik może dodać pewną kompletnie nieistniejącą w systemie funkcjonalność do poprawy działania systemu, czy automatyzacji innych rzeczy niezwiązanych z domem.

% https://analytics.home-assistant.io/

\subsection{AppDaemon -- AD}
System pozwalający stworzenie modułu samouczącego powinien dawać maksimum możliwości i niezależności, zatem wybrano AppDaemon. AppDaemon to środowisko wykonawczego Pythona, w wielowątkowej architekturze piaskownicy, służące do pisania aplikacji automatyzacji dla projektów automatyki domowej (i nie tylko), których wymogiem jest solidna architektura sterowana zdarzeniami. AD jest od razu gotowe do współpracy z systemem HomeAssistant, co sprawia, że integracja systemów HA/AD jest bezproblemowa. System pozwala natychmiastowo reagować na zmiany stanów urządzeń znajdujących się w domowym środowisku, poprzez korzystanie z asynchronicznej architektury callback.
% Wybór HA jako systemu nadzorczego ogranicza nam pole wyboru rozwiązań które

\subsection{Docker}
Ze względu na wymóg niezależności od platformy na jakiej będzie znajdować się ten system, zdecydowano o wyborze jako jednej z głównych technologii, systemu Docker w celu zapewnienia aplikacjom pracującym pod nim odpowiednich warunków niezależnie od systemu operacyjnego na jakim się znajduje. Dodatkowym atutem, który sprawił że wybrano 

\subsection{Tensorflow}
Wszystkie moje ziomki korzystają z tf.

\subsection{Python}
Wykorzystanie struktury HA/AD ogranicza nas do wyboru Python jako przewodniego języka programowania w tym projekcie.