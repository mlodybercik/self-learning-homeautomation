% !TEX encoding = UTF-8 Unicode 

\chapter{Wstęp}

\section{Wprowadzenie do problematyki}

Automatyzacja to określenie na metody mające na celu ograniczenie ludzkiej interakcji do minimum w różnych procesach. Stosuje się ją w wielu dziedzinach począwszy na przemyśle, kończąc na procesach informatycznych. Pewnym jednym konkretnym obszarem zyskującym w ostatnim czasie dużo zainteresowania jest pojęcie inteligentnego domu (smart-home). Automatyka domowa to konkretne określenie na zastosowanie automatyzacji wśród urządzeń smart-home skupiających się na obszarze zastosowań gospodarstwa domowego. Celem takiego zastosowania jest kontrola pracy urządzeń znajdujących się w domu do osiągnięcia konkretnego ich stanu w sposób minimalizujący ludzką interakcję. W praktyce takie systemy to bardzo dobre rozwiązanie które niesie ze sobą wiele zalet i uproszczeń w codziennym życiu \cite{szablowski2019projektowanie}.
% https://automatykab2b.pl/temat-miesiaca/39129-automatyka-budynkowa-czesc-1?show=1

Zasada czy reguła w automatyce to w dokładny sposób określenie jakie akcje muszą zostać wykonane pod warunkiem pewnego stanu systemu. Warunkiem początkowym może być każdy stan dowolnego urządzenia w systemie, zmiana stanu dowolnego urządzania lub całkowicie zewnętrznego bodźca. W przypadku takiej automatyzacji często też korzysta się z kombinacyjnego połączenia wielu źródeł i informacji w celu stworzenia stanu początkowego, który jeśli wystąpi, jest przesłanką do wysłania przez system zarządzający urządzeniami polecenia do wykonania pojedynczej lub ciągu akcji. Każde dostępne na rynku dedykowane oprogramowanie obsługujące automatykę domową dostarcza różne narzędzia do szybkiego i intuicyjnego sporządzania takich zasad ale także dostarcza sposoby na komunikację z tymi urządzeniami. Mamy zatem pełny system, który agreguje wiele systemów, protokołów i sposobów wymiany danych w celu monitorowania i sterowania urządzeniami w ramach (lokalnej) sieci.

\section{Opis problemu}

Mimo tego, że analiza i projektowanie systemów inteligentnej automatyki domów zaczęła się kilka dekad temu, istnieje wiele nierozwiązanych problemów, które muszą zostać rozwiązane, aby taka forma stała się popularna. Problemami są nieprawidłowe algorytmy oraz niskie celności wynikowe \cite{episode_discovery_2}.

Mimo wygody w używaniu i łatwości w sporządzaniu zasad automatyki domowych zastosowań pojawiają się pewne trudności. Jednym z nich jest problem ze zmiennością ludzkich nawyków. Sporządzenie zasad sprawia, że dane czynności są wykonywane wtedy i tylko wtedy gdy zostanie spełniony pewien konkretny stan określony przez nas podczas sporządzania danej reguły. Z logicznego punktu widzenia jest to odpowiednia odpowiedź systemu, natomiast z tego praktycznego już nie, ponieważ dana reguła może zostać wykonana mimo tego, że nie chcieliśmy aby się wykonała. Dodaje to pewien wymóg dodatkowego komplikowania danej zasady poprzez dodatkowe warunki w systemie lub ciągłe jej edytowanie aby odpowiadała naszym zmiennym nawykom.

Bez względu na to jakie narzędzia dostarcza nam dany system, sporządzanie dużej ilości skomplikowanych reguł może być problematyczne. Same środowiska zarządzania takimi systemami mimo wspierania tworzenia na tyle skomplikowanych zasad mogą nie być do tego przystosowane. Często zdarza się, że system automatyki udostępnia dodatkowe, jako wtyczki lub dodatki, moduły do tworzenia reguł za pomocą interpretowanych języków programowania \cite{appdaemon:main}, \cite{domoticz:scripts}, \cite{openhab:scripts}. Tworzy to natomiast zestaw innych problemów, gdzie użytkownik chcący stworzyć takie reguły, musi w przynajmniej podstawowy sposób znać języki języki programowania oraz dodatkowo w jaki sposób sporządzać te skrypty aby mogły być wykonywane przez oprogramowanie nadzorujące.



\section{Cel pracy}
Celem niniejszej pracy jest opracowanie modułu do systemu automatyki domowej (smart-home) wspierającego tworzenie reguł decyzyjnych w oparciu o mechanizmy uczące się zachowań użytkownika.

% - Przegląd literatury w zakresie budowy urządzeń Internetu Rzeczy.
% To mi się nie podoba giga, bede musiał zagadać o to.

\section{Zakres pracy}
Zakres pracy obejmuje przegląd literatury w zakresie działania i budowy systemów wspomagania podejmowania decyzji podejmowania decyzji opartych o uczenie maszynowe.

% Został przetestowany w następujących narzędziach:
% \begin{itemize}
% \item Overleaf.com -- wersja on-line; nie jest wymagana instalacja
% \item TeXStudio/TeXLive oraz TeXStudio/MiKTeX oraz TeXWorks/MiKTex
% \end{itemize}

% Użycie pełnej wersji może wymagać (np. dla zestawu TeXWorks/MikTeX):
% \begin{itemize}
% \item instalacji Python 2.7+ wraz z pakietem Pygments; ścieżki do obu narzędzi powinny być ustawione w zmiennej środowiskowej PATH -- pakiet jest używany do kolorowania słów kluczowych w listingach języków programowania
% \item w środowisku Latex należy włączyć opcję -shell-escape  -- jest wymagana dla pakietu ``minted'' Latexa 
% \end{itemize}

% \section*{Cel pracy}

% \lipsum[5]

% \section*{Zakres pracy}

% \lipsum[6]