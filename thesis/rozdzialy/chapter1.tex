% !TEX encoding = UTF-8 Unicode 

\chapter{Wstęp}

\section{Wprowadzenie do problematyki}

Automatyzacja to określenie na metody mające na celu ograniczenie ludzkiej interakcji do minimum w różnych procesach. Stosuje się ją w wielu dziedzinach począwszy na przemyśle, kończąc na procesach informatycznych. Pewnym jednym konkretnym obszarem zyskującym w ostatnim czasie dużo zainteresowania jest pojęcie inteligentnego domu (smart-home). Automatyka domowa to konkretne określenie na zastosowanie automatyzacji wśród urządzeń smart-home skupiających się na obszarze zastosowań gospodarstwa domowego. Celem takiego zastosowania jest kontrola pracy urządzeń znajdujących się w domu do osiągnięcia konkretnego ich stanu w sposób minimalizujący ludzką interakcję. W praktyce takie systemy to bardzo dobre rozwiązanie które niesie ze sobą wiele zalet i uproszczeń w codziennym życiu \cite{szablowski2019projektowanie}.
% https://automatykab2b.pl/temat-miesiaca/39129-automatyka-budynkowa-czesc-1?show=1

Reguła w automatyce, czasem nazywana zasadą, to w dokładny sposób określenie jakie akcje muszą zostać wykonane pod warunkiem pewnego stanu systemu. Warunkiem początkowym może być każdy stan, zmiana stanu dowolnego urządzania lub całkowicie zewnętrzny bodziec. W przypadku takiej automatyzacji często też korzysta się z kombinacyjnego połączenia wielu źródeł i informacji w celu stworzenia stanu początkowego, który jeśli wystąpi, jest przesłanką do wysłania przez system zarządzający urządzeniami polecenia do wykonania pojedynczej lub ciągu akcji. Każde dostępne na rynku dedykowane oprogramowanie obsługujące automatykę domową dostarcza różne narzędzia do szybkiego i intuicyjnego sporządzania takich zasad, ale także dostarcza sposoby na komunikację z tymi urządzeniami. Dostepny jest zatem pełny system, który agreguje wiele pomniejszych mechanizmów, protokołów i sposobów wymiany danych w celu monitorowania i sterowania urządzeniami w ramach (lokalnej) sieci.

\section{Opis problemu}

Mimo tego, że analiza i projektowanie systemów inteligentnej automatyki domów zaczęła się kilka dekad temu, istnieje wiele nierozwiązanych problemów, które muszą zostać rozwiązane, aby autonomiczna forma stała się popularna. Ograniczeniami są nieprawidłowe algorytmy oraz niskie celności wynikowe \cite{episode_discovery_2}.

Mimo wygody w używaniu i łatwości w sporządzaniu zasad automatyki domowej pojawiają się pewne trudności. Jednym z nich jest problem ze zmiennością ludzkich nawyków. Sporządzenie zasad sprawia, że dane czynności zostaną wykonywane wtedy i tylko wtedy gdy zostanie spełniony pewien konkretny stan określony przez użytkownika podczas sporządzania danej reguły. Z logicznego punktu widzenia jest to odpowiednia reakcja systemu, natomiast z tego praktycznego już nie, ponieważ dana reguła może zostać wykonana mimo tego, że w pewnych specyficznych okolicznościach użytkownik nie zechce aby się wykonała. Dodaje to pewien wymóg dalszego komplikowania danej zasady poprzez dodatkowe warunki w systemie lub ciągłe jej edytowanie aby odpowiadała naszym zmiennym nawykom.

Bez względu na to, jakie narzędzia dostarcza nam dany system, sporządzanie dużej liczby skomplikowanych reguł może być problematyczne. Same środowiska zarządzania takimi systemami mimo wspierania tworzenia na tyle skomplikowanych zasad mogą nie być do tego przystosowane. Często zdarza się, że system automatyki udostępnia dodatkowe, jako wtyczki lub dodatki, moduły do tworzenia reguł za pomocą interpretowanych języków programowania \cite{appdaemon:main}, \cite{domoticz:scripts}, \cite{openhab:scripts}. Tworzy to natomiast zestaw innych problemów, gdzie użytkownik chcący stworzyć takie reguły, musi w przynajmniej podstawowy sposób znać języki programowania, a ponadto mieć wiedzę w jaki sposób sporządzać te skrypty aby mogły być wykonywane przez oprogramowanie nadzorujące.


\section{Cel pracy}
Celem niniejszej pracy jest opracowanie modułu do systemu automatyki domowej (smart-home) wspierającego tworzenie reguł decyzyjnych w oparciu o mechanizmy uczące się zachowań użytkownika.


\section{Zakres pracy}
Zakres pracy obejmuje przegląd literatury w obszarze działania i budowy systemów wspomagania podejmowania decyzji opartych o uczenie maszynowe, a także przegląd tematyki związanej z działaniem i budową urządzeń Internetu Rzeczy.
